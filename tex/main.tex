\documentclass[12pt]{report}
\linespread{1.3}

\usepackage[utf8]{inputenc}
\usepackage[T1]{fontenc}
\usepackage{xcolor}
\usepackage{inconsolata} % font type

\usepackage{lmodern}
\usepackage{graphicx}
% \graphicspath{ {figure/} }

\usepackage{t1enc}

\definecolor{pblue}{rgb}{0.13,0.13,1}
\definecolor{pgreen}{rgb}{0,0.5,0}
\definecolor{pred}{rgb}{0.9,0,0}
\definecolor{pgrey}{rgb}{0.46,0.45,0.48}
\definecolor{plightgrey}{rgb}{0.94,0.94,0.96}
\definecolor{pmagenta}{rgb}{0.15,0.37,0.46}
\definecolor{pgreen2}{rgb}{0.15,0.37,0.46}

\usepackage{listings}
\lstset{
  captionpos=b,
  showspaces=false,
  showtabs=false,
  breaklines=true,
  showstringspaces=false,
  breakatwhitespace=true,
  commentstyle=\color{pgreen},
  keywordstyle=\color{pblue},
  stringstyle=\color{pred},
  basicstyle=\ttfamily,
  moredelim=[il][\textcolor{pgrey}]{$$},
  moredelim=[is][\textcolor{pgrey}]{\%\%}{\%\%},
  basicstyle=\footnotesize,
  frame=single,
  numbers=left,
  stepnumber=1,
  showstringspaces=false,
  tabsize=1,
  breaklines=true,
  breakatwhitespace=false,
  backgroundcolor=\color{plightgrey}
}

\lstset{basicstyle=\ttfamily\footnotesize,breaklines=true}

% \usepackage{csquotes}
% \usepackage[magyar]{babel}
\usepackage[backend=biber]{biblatex}
\usepackage{geometry}
\geometry{
 a4paper,
 total={160mm,247mm},
 left=25mm,
 top=25mm,
}

\usepackage{titlesec}
\newcommand{\hsp}{\hspace{15pt}}
\titleformat{\chapter}[hang]{\normalfont\Huge\bfseries}{\arabic{chapter} \hsp\color{lightgray}|\hsp}{1ex}{}
\titlespacing*{\chapter}{0pt}{-30pt}{30pt}

\usepackage{hyperref}

\bibliography{ref.bib}

\begin{document}

\thispagestyle{empty}
\begin{center}

{\Huge\bf Data Warehouses and Big Data\newline Final project}\\
\vspace{2.5cm}
{\Large\bf Attila Ambrus} \\
\vspace{0.5cm}
{\small Computer Scientist\\BSc student}\\
\vspace{1cm}
Coimbra, 2024
\vspace{1.5cm}


\end{center}


\thispagestyle{empty}
\clearpage
\pagenumbering{arabic}

\chapter{Introduction}
A data warehouse is a centralized repository designed to store, manage, and analyze large volumes of structured and semi-structured data from multiple sources. Unlike traditional databases, which are optimized for transaction processing, data warehouses are optimized for querying and reporting, making them an essential component in business intelligence and analytics.

The primary function of a data warehouse is to provide a cohesive and consistent data source for decision-making processes. It integrates data from various operational systems, such as customer relationship management (CRM), enterprise resource planning (ERP), and other internal and external data sources, transforming it into a unified format. This integration enables organizations to perform complex queries and generate reports, dashboards, and visualizations that support strategic planning and operational improvements.

Data warehouses employ a variety of techniques to ensure data integrity, accuracy, and accessibility. These techniques include data cleaning, transformation, and loading (ETL), as well as the use of schemas such as star and snowflake to organize data in a way that optimizes performance. Modern data warehouses also leverage advanced technologies such as cloud computing, in-memory processing, and machine learning to enhance scalability, speed, and analytical capabilities.

The evolution of data warehousing has paralleled advancements in technology, leading to the development of data lakes and hybrid models that can handle unstructured data and support real-time analytics. As organizations increasingly rely on data-driven insights to gain a competitive edge, the role of data warehouses continues to expand, making them a cornerstone of modern data architecture.

\chapter{Tools}

\section{Docker}

Docker is an open-source platform designed to automate the deployment, scaling, and management of applications using containerization. Containers are lightweight, portable units that package an application and its dependencies together, ensuring consistent behavior across different environments. Docker simplifies the process of creating, deploying, and running applications by providing a standardized unit of software. This allows developers to build, test, and deploy applications quickly and efficiently, irrespective of the underlying infrastructure. Docker's core components include the Docker Engine, which runs containers, and Docker Hub, a repository for sharing container images. It is widely used in modern DevOps practices for its ability to streamline development workflows, enhance scalability, and improve resource utilization.

\section{ERDPlus}

ERDPlus is an online tool for creating entity-relationship diagrams (ERDs), relational schemas, star schemas, and other database design diagrams. It is user-friendly and provides a drag-and-drop interface, making it easy for users to visually design and document their database structures. ERDPlus supports the creation of detailed diagrams that represent the data models and relationships within a database, facilitating better understanding and communication among developers, analysts, and stakeholders. This tool is particularly useful for designing databases from scratch, planning database modifications, and ensuring that the database design aligns with business requirements.

\section{PostgreSQL}

PostgreSQL, often referred to as Postgres, is a powerful, open-source relational database management system (RDBMS). Known for its robustness, extensibility, and standards compliance, PostgreSQL supports a wide range of data types and advanced features such as transactions, foreign keys, subqueries, triggers, and views. It also includes support for JSON and XML, allowing it to handle both structured and semi-structured data. PostgreSQL is highly customizable and extensible, offering numerous extensions for full-text search, geospatial data (PostGIS), and more. Its strong community support and comprehensive documentation make it a popular choice for developers and organizations needing a reliable and versatile database solution.

\section{pgAdmin}

pgAdmin is an open-source management and administration tool for PostgreSQL. It provides a graphical interface that simplifies the tasks of database administration, including creating databases, running queries, managing users and permissions, and monitoring database performance. pgAdmin supports multiple PostgreSQL versions and offers features like SQL query editor, debugging tools, and data visualization capabilities. It is available as a desktop application and a web-based tool, making it accessible from various platforms and devices. pgAdmin is essential for database administrators and developers who require a user-friendly interface to interact with PostgreSQL databases.

\section{Radzen}

Radzen is a low-code, rapid application development platform that enables users to build web applications with minimal coding effort. It provides a visual interface for designing user interfaces, integrating with various data sources, and automating workflows. Radzen supports the creation of Angular and Blazor applications and can connect to a variety of databases, including SQL Server, Oracle, MySQL, and PostgreSQL. It offers drag-and-drop components, code generation, and customization options, allowing developers to quickly prototype and deploy applications. Radzen is particularly useful for business applications where time-to-market and ease of maintenance are critical.

\section{DbSchema}

DbSchema is a comprehensive database design and management tool that supports various relational and NoSQL databases, including MySQL, PostgreSQL, MongoDB, SQL Server, and more. It provides a visual interface for designing schemas, creating and executing queries, and managing database objects. DbSchema includes features such as interactive diagrams, data exploration, synchronization, and documentation generation. It also offers capabilities for reverse-engineering existing databases and visual query builders. DbSchema is valuable for developers, database administrators, and data architects who need a versatile and powerful tool to manage and document their database environments effectively.

%s/\(.*,\)".*\d \(.*\)"/\1"\2"

% \printbibliography
% \addcontentsline{toc}{chapter}{References}

\end{document}
